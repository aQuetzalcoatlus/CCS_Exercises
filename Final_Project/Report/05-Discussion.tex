This project focused on the analysis of an alanine dipeptide simulation data using the methods of PCA, dPCA, and dPCA+. After an overview on the nature of the data points, its periodicity and time evolution, three analysis techniques were employed to analyze the dihedral angle data.

The PCA technique highlights the overall conformations present in the molecule. Although it was straightforward to implement, in the context of analysis of backbone dihedral angles, traditional PCA struggles with periodic metrics, as it's primarily designed for linear data. Both covariance and variance calculations, which are central to PCA, become flawed.

In contrast, dPCA transforms these angles using sine and cosine functions, effectively linearizing this circular data, preserving the true relationships and providing a more accurate analysis. This transformation addresses the core issue of the undefined mean in a circular context, ensuring that the resulting analysis represents the actual dynamics of the molecule in study. Although the minima of the free energy landscape looked more sharper to distinguish, they may give rise to complicated pattern, which are not so easy to interpret into the actual conformational states of the molecule. 

Finally, the dPCA+ technique is designed to preserve the torus topology, and uses a linear transformation that minimizes periodicity-induced errors in covariance estimation and avoids artificial extra dimensions or distortions in probability distribution \cite{stock2017}. This was a comparatively simple algorithm to implement, with the only restriction that the data under consideration should have regions of maximal gap present in them. 